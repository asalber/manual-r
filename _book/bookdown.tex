% Options for packages loaded elsewhere
\PassOptionsToPackage{unicode}{hyperref}
\PassOptionsToPackage{hyphens}{url}
%
\documentclass[
]{book}
\usepackage{amsmath,amssymb}
\usepackage{lmodern}
\usepackage{iftex}
\ifPDFTeX
  \usepackage[T1]{fontenc}
  \usepackage[utf8]{inputenc}
  \usepackage{textcomp} % provide euro and other symbols
\else % if luatex or xetex
  \usepackage{unicode-math}
  \defaultfontfeatures{Scale=MatchLowercase}
  \defaultfontfeatures[\rmfamily]{Ligatures=TeX,Scale=1}
\fi
% Use upquote if available, for straight quotes in verbatim environments
\IfFileExists{upquote.sty}{\usepackage{upquote}}{}
\IfFileExists{microtype.sty}{% use microtype if available
  \usepackage[]{microtype}
  \UseMicrotypeSet[protrusion]{basicmath} % disable protrusion for tt fonts
}{}
\makeatletter
\@ifundefined{KOMAClassName}{% if non-KOMA class
  \IfFileExists{parskip.sty}{%
    \usepackage{parskip}
  }{% else
    \setlength{\parindent}{0pt}
    \setlength{\parskip}{6pt plus 2pt minus 1pt}}
}{% if KOMA class
  \KOMAoptions{parskip=half}}
\makeatother
\usepackage{xcolor}
\IfFileExists{xurl.sty}{\usepackage{xurl}}{} % add URL line breaks if available
\IfFileExists{bookmark.sty}{\usepackage{bookmark}}{\usepackage{hyperref}}
\hypersetup{
  pdftitle={Curso básico de análisis de datos con R},
  pdfauthor={Alfredo Sánchez Alberca},
  hidelinks,
  pdfcreator={LaTeX via pandoc}}
\urlstyle{same} % disable monospaced font for URLs
\usepackage{longtable,booktabs,array}
\usepackage{calc} % for calculating minipage widths
% Correct order of tables after \paragraph or \subparagraph
\usepackage{etoolbox}
\makeatletter
\patchcmd\longtable{\par}{\if@noskipsec\mbox{}\fi\par}{}{}
\makeatother
% Allow footnotes in longtable head/foot
\IfFileExists{footnotehyper.sty}{\usepackage{footnotehyper}}{\usepackage{footnote}}
\makesavenoteenv{longtable}
\usepackage{graphicx}
\makeatletter
\def\maxwidth{\ifdim\Gin@nat@width>\linewidth\linewidth\else\Gin@nat@width\fi}
\def\maxheight{\ifdim\Gin@nat@height>\textheight\textheight\else\Gin@nat@height\fi}
\makeatother
% Scale images if necessary, so that they will not overflow the page
% margins by default, and it is still possible to overwrite the defaults
% using explicit options in \includegraphics[width, height, ...]{}
\setkeys{Gin}{width=\maxwidth,height=\maxheight,keepaspectratio}
% Set default figure placement to htbp
\makeatletter
\def\fps@figure{htbp}
\makeatother
\setlength{\emergencystretch}{3em} % prevent overfull lines
\providecommand{\tightlist}{%
  \setlength{\itemsep}{0pt}\setlength{\parskip}{0pt}}
\setcounter{secnumdepth}{5}
\usepackage{booktabs}
\ifLuaTeX
  \usepackage{selnolig}  % disable illegal ligatures
\fi
\usepackage[]{natbib}
\bibliographystyle{plainnat}

\title{Curso básico de análisis de datos con R}
\author{Alfredo Sánchez Alberca}
\date{2021-12-24}

\begin{document}
\maketitle

{
\setcounter{tocdepth}{1}
\tableofcontents
}
\hypertarget{prefacio}{%
\chapter*{Prefacio}\label{prefacio}}
\addcontentsline{toc}{chapter}{Prefacio}

\hypertarget{propuxf3sito-de-este-manual}{%
\section*{Propósito de este manual}\label{propuxf3sito-de-este-manual}}
\addcontentsline{toc}{section}{Propósito de este manual}

Este manual proporciona una introducción amigable al \href{https://www.r-project.org/}{lenguaje de programación R} para aquellas personas interesadas en utilizar este lenguaje para el análisis de datos.

El manual empieza con los conceptos básicos del lenguaje de programación R pero enseguida aborda su uso para la visualización y el análisis estadístico de datos, haciendo un recorrido por los test estadísticos más comunes.

Lo más interesante de este manual es la multitud de ejemplos que ilustran el uso de las técnicas estadísticas presentadas, así como los problemas propuestos.

El manual no aborda los fundamentos matemáticos de los análisis estadísticos presentados, aunque si explica brevemente cuándo deben usarse y cuándo no, así como las interpretaciones de los resultados obtenidos en los ejemplos. Si alguien está interesado en profundizar en los detalles matemáticos, puede visitar esta \href{https://aprendeconalf.es/docencia/estadistica/}{página}.

No es un curso de programación en R, sino de uso de sus funciones predefinidas y de los paquetes más habituales para el análisis de datos.

Para cualquier comentario o sugerencia sobre este manual escriba al autor (\href{mailto:asalber@ceu.es}{\nolinkurl{asalber@ceu.es}}).

\hypertarget{licencia}{%
\section*{Licencia}\label{licencia}}
\addcontentsline{toc}{section}{Licencia}

Este trabajo se libera bajo licencia \href{https://creativecommons.org/licenses/by-sa/4.0/deed.es}{Creative Commons Atribución-CompartirIgual 4.0 (CC BY-SA 4.0)}

Manual básico de análisis de datos con R by Alfredo Sánchez Alberca is licensed under CC BY-SA 4.0

\hypertarget{introducciuxf3n-a-r}{%
\chapter{Introducción a R}\label{introducciuxf3n-a-r}}

La gran potencia de cómputo alcanzada por los ordenadores ha convertido a los mismos en poderosas herramientas al
servicio de todas aquellas disciplinas que, como la Estadística, requieren manejar un gran volumen de datos.
Actualmente, prácticamente nadie se plantea hacer un estudio estadístico serio sin la ayuda de un buen programa de
análisis de datos.

\emph{R} es un potente lenguaje de programación que incluye multitud de funciones para la representación y el análisis de
datos.
Fue desarrollado por Robert Gentleman y Ross Ihaka en la Universidad de Auckland en Nueva Zelanda, aunque actualmente es mantenido por una enorme comunidad científica en todo el mundo.

\begin{figure}

{\centering \includegraphics[width=0.25\linewidth]{img/Rlogo} 

}

\caption{Logotipo de R}\label{fig:rlogo}
\end{figure}

Las ventajas de R frente a otros programas habituales de análisis de datos, como pueden ser SPSS, SAS o Matlab, son múltiples:

\begin{itemize}
\tightlist
\item
  Es software libre y por tanto gratuito. Puede descargarse desde la web \url{http://www.r-project.org/}.
\item
  Es multiplataforma. Existen versiones para Windows, Macintosh, Linux y otras plataformas.
\item
  Está avalado y en constante desarrollo por una amplia comunidad científica distribuida por todo el mundo que lo utiliza como estándar para el análisis de datos.
\item
  Cuenta con multitud de paquetes para todo tipo de análisis estadísticos y representaciones gráficas, desde los más
  habituales, hasta los más novedosos y sofisticados que no incluyen otros programas. Los paquetes están organizados y
  documentados en un \href{https://cran.r-project.org/}{repositorio CRAN} (Comprehensive R Archive Network) desde donde pueden descargarse libremente.
\item
  Es programable, lo que permite que el usuario pueda crear fácilmente sus propias funciones o paquetes para
  análisis de datos específicos.
  Existen multitud de libros, manuales y tutoriales libres que permiten su aprendizaje e ilustran el análisis
  estadístico de datos en distintas disciplinas científicas como las Matemáticas, la Física, la Biología, la Psicología, la Medicina, etc.
\end{itemize}

\hypertarget{entornos-de-desarrollo}{%
\section{Entornos de desarrollo}\label{entornos-de-desarrollo}}

Por defecto el entorno de trabajo de R es en línea de comandos, lo que significa que los cálculos y los análisis se realizan mediante comandos o instrucciones que el usuario teclea en una ventana de texto. No obstante, existen distintas
interfaces gráficas de usuario que facilitan su uso, sobre todo para usuarios noveles. Algunas de ellas, como las que se enumeran a continuación, son completos entornos de desarrollo que facilitan la gestión de cualquier proyecto:

\begin{itemize}
\tightlist
\item
  \href{https://www.rstudio.com/}{RStudio}. Probablemente el entorno de desarrollo más extendido para programar con R ya que incorpora multitud de utilidades para facilitar la programación con R.
\item
  \href{https://rkward.kde.org}{RKWard}. Es otra otro de los entornos de desarrollo más completos que además incluye a posibilidad de añadir nuevos menús y cuadros de diálogo personalizados.
\item
  \href{https://code.visualstudio.com/}{Visual Studio Code}. Es un entorno de desarrollo de propósito general ampliamente extendido. Aunque no es un entorno de desarrollo específico para R, incluye una extensión con utilidades que facilitan mucho el desarrollo con R.
\end{itemize}

\hypertarget{estructuras-de-control}{%
\chapter{Estructuras de control}\label{estructuras-de-control}}

\hypertarget{funciones}{%
\chapter{Funciones}\label{funciones}}

\hypertarget{gruxe1ficos-y-visualizaciuxf3n-de-datos}{%
\chapter{Gráficos y visualización de datos}\label{gruxe1ficos-y-visualizaciuxf3n-de-datos}}

\hypertarget{estaduxedstica-descriptiva}{%
\chapter{Estadística descriptiva}\label{estaduxedstica-descriptiva}}

\hypertarget{estimaciuxf3n-de-paruxe1metros-y-contrastes-de-hipuxf3tesis}{%
\chapter{Estimación de parámetros y contrastes de hipótesis}\label{estimaciuxf3n-de-paruxe1metros-y-contrastes-de-hipuxf3tesis}}

  \bibliography{bibliography.bib}

\end{document}
